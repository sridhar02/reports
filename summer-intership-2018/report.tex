\documentclass[11pt,a4paper]{article}
\usepackage{fontspec}
\usepackage{multirow}
\defaultfontfeatures{LetterSpace=5}
\setmainfont{Times New Roman}
\usepackage{graphicx}
\usepackage{setspace}
\usepackage[a4paper,right=1in,left=1.5in,top=1in,bottom=1in]{geometry}

% TODO Use this package to link from Contents and Sections
% \usepackage[colorlinks=true,linkcolor=blue]{hyperref}

\makeatletter

\newcommand\frontmatter{%
    \cleardoublepage
  %\@mainmatterfalse
  \pagenumbering{roman}}

\newcommand\mainmatter{%
    \cleardoublepage
 % \@mainmattertrue
  \pagenumbering{arabic}}

\newcommand\backmatter{%
  \if@openright
    \cleardoublepage
  \else
    \clearpage
  \fi
 % \@mainmatterfalse
   }

\makeatother

\begin{document}

\frontmatter

\begin{titlepage}
\begin{center}

\textbf{A Summer Internship Report}

\vspace{1em}

\large \textbf{OVERVIEW OF OFFSHORE TERMINAL OF RIL AT GADIMOGA}

\vspace{3em}

\large \textit{\textbf{A report submitted in partial fulfilment for the award of degree of}}

\vspace{1em}

\textbf{Bachelor of Technology in Petroleum Engineering}

\vspace{3em}

\textit{By}

\vspace{1em}

\textbf{Student Name}
    
\textbf{(Roll no.)}

\vspace{2em}

\textit{Under the Guidance and Supervision of}

\vspace{1em}

\textbf{Guide Name}

\textbf{Designation of Guide}

\textbf{Name of the Industry}

\vspace{2em}

\includegraphics[scale=0.3]{jntuk1}

\vspace{1em}

\doublespacing

\large \textbf{DEPARTMENT OF PETROLEUM ENGINEERING AND \\ 
PETROCHEMICAL ENGINEERING}

\large \textbf{UNIVERSITY COLLEGE OF ENGINEERING KAKINADA (A)}

\large \textbf{JAWAHARLAL NEHRU TECHNOLOGICAL UNIVERSITY KAKINADA}

\large \textbf{KAKINADA – 533003}

\large \textbf{2018}

\end{center}
\end{titlepage}

\newpage

\begin{center}
\section*{DECLARATION}
\end{center}

\vspace{4em}


I, \textbf{KATTA SRIDHAR}, hereby declare that this summer internship report entitled “Drilling Services” is original and has not previously formed the basis for the award of any degree to similar work.


\vspace{5em}

\noindent Place: KAKINADA  \hfill Signature     \hspace{0.02\textwidth}

\vspace{1em}

\noindent DATE: 1-12-2018  \hfill Katta Sridhar


\newpage

\begin{center}
\section*{AKNOWLEDGEMENTS}
\end{center}


\vspace{1em}

I would like to express my profound sense of gratitude to my guide and supervisor, Name, \textbf{Designation, Department, Industry,} for his skillful guidance, timely suggestions and encouragement in completing this project. (Others in industry who helped in the internship program may be mentioned)

\vspace{1em}


I acknowledge my sincere thanks and deep felt gratitude to \textbf{Prof. K.V.Rao}, Programme Director, Petroleum Courses, Jawaharlal Nehru Technological University Kakinada for arranging the summer internship program.

\vspace{1em}

I take this opportunity to express my sincere thanks to \textbf{Dr. K. Meera Saheb}, Head of the Department, Department of Petroleum Engineering and Petrochemical Engineering for encouraging and motivating me to complete the internship and report successfully.

\vspace{1em}

Also I express my special thanks to our beloved Principal, \textbf{Dr. P. Subba Rao} his enthusiastic support in our endeavours.

\vspace{1em}

Finally, I am very much grateful to my parents for their financial support and encouragement throughout the internship programme.

\vspace{1em}

\hfill \textbf{Katta Sridhar}

% TODO Center Roll Number w.r.t Name  
\hfill \textbf{Roll number} \hspace{0.005\textwidth}
        
\newpage        
\tableofcontents



\newpage

\begin{center}
\section*{ABSTRACT}
\end{center}
  
The abstract is to be in fully-justified italicized text, at the top of the left-hand column as it is here, below the author information. Use the word “Abstract” as the title, in 12-point Times, boldface type, centered relative to the column, initially capitalized. The abstract is to be in 10-point, single-spaced type, and may be up to 3 in. (7.62 cm) long. Leave two blank lines after the abstract, then begin the main text. All manuscripts must be in English.
  
\vspace{2em}
  
An Abstract is required for every paper; it should succinctly summarize the reason for the work, the main findings, and the conclusions of the study. The abstract should be no longer than 250 words. Do not include artwork, tables, elaborate equations or references to other parts of the paper or to the reference listing at the end. The reason is that the Abstract should be understandable in itself to be suitable for storage in textual information retrieval systems.
  
\newpage


\listoftables


\newpage


\listoffigures


\newpage

\mainmatter

\newpage



\chapter{About ONGC}


Oil and Natural Gas Corporation Limited (ONGC) is an Indian multinational oil and gas company earlier headquartered in Dehradun, Uttarakhand, India. As a Corporation, its registered office is now at Deendayal Uurja Bhavan, Vasant Kunj, New Delhi (110070) India. It is a Public Sector Undertaking (PSU) of the
Government of India, under the administrative control of the Ministry of Petroleum and Natural Gas. It is India's largest oil and gas exploration and production company. It produces around 70\% of India's crude oil (equivalent to around 30\% of the country's total demand) and around 62\% of its natural gas.

\section{History of ONGC}

1. Foundation to 1961

a) Before the independence of India in 1947, the Assam Oil Company in the north-eastern and Attock Oil Company in north-western part of the undivided India
were the only oil producing companies, with minimal exploration input. The major part of Indian sedimentary basins was deemed to be unfit for development of oil
and gas resources.

b) After independence, the Central Government of India realized the importance of oil and gas for rapidindustrial development and its strategic role in defense.
Consequently, while framing the Industrial Policy Statement of 1948, the development of petroleum industry in the country was considered to be of utmost
necessity.

c) In 1955, Government of India decided to develop the oil and natural gas resources in the various regions of the country as part of the Public Sector development. With this objective, an Oil and Natural Gas Directorate was set up towards the end of 1955, as a subordinate office under the then Ministry of Natural Resources and
Scientific Research. The department was constituted with a nucleus of geoscientists from the Geological Survey of India.

d) Soon, after the formation of the Oil and Natural Gas Directorate, it became apparent that it would not be possible for the Directorate with its limited financial and administrative powers as subordinate office of the Government, to function efficiently. So in August, 1956, the Directorate was raised to the status of a commission with enhanced powers, although it continued to be under the government. In October 1959, the Commission was converted into a statutory body by an act of the Indian Parliament, which enhanced powers of the commission further. The main functions of the Oil and Natural Gas Commission subject to the provisions of the Act, were "to plan, promote, organize and implement programs for development of Petroleum Resources and the production and sale of petroleum and petroleum products produced by it, and to perform such other functions as the Central Government may, from time to time, assign to it ". The act further outlined the activities and steps to be taken by ONGC in fulfilling its mandate.

2. From 1961 to present

a) Since its inception, ONGC has been instrumental in transforming the country's limited upstream sector into a large viable playing field, with its activities spread throughout India and significantly in overseas territories. In the inland areas, ONGC not only found new resources in Assam but also established new oil province in Cambay basin (Gujarat), while adding new petroliferous areas in the Assam-Arakan Fold Belt and East coast basins (both onshore and offshore).

b) ONGC became a publicly held company in February 1994, with 20\% of its equity were sold to the public and eighty percent retained by the Indian government. At the time, ONGC employed 48,000 people and had reserves and surpluses worth ₹104.34 billion, in addition to its intangible assets. The corporation's net worth of ₹107.77 billion was the largest of any Indian company .

c) In 2003, ONGC Videsh Limited (OVL), the division of ONGC concerned with its foreign assets, acquired Talisman Energy's 25\% stake in the Greater Nile Oil project.

d) In 2006, a commemorative coin set was issued to mark the 50th anniversary of the founding of ONGC, making it only the second Indian company (State Bank of India being the first) to have such a coin issued in its honor.

e) In 2011, ONGC applied to purchase 2000 acres of land at Dahanu to process offshore gas. ONGC Videsh, along with Statoil ASA (Norway) and Repsol SA (Spain), has been engaged in deep-water drilling off the northern coast of Cuba in 2012. On 11 August 2012, ONGC announced that it had made a large oil discovery in the D1 oilfield off the west coast of India, which will help it to raise the output of the field from around 12,500 barrels per day (bpd) to a peak output of 60,000 bpd.

f) In November 2012, OVL agreed to acquire ConocoPhillips' 8.4\% stake in the Kashagan oilfield in Kazakhstan for around US\$5 billion, in ONGC's largest acquisition to date. The acquisition is subject to the approval of the governments of Kazakhstan and India and also to other partners in the Caspian Sea field waiving their pre-emption rights.

g) In January 2014, OVL and Oil India completed the acquisition of Videocon Group's ten percent stake in a Mozambican gas field for a total of \$2.47 billion.

h) In June 2015, Oil and Natural Gas Corporation (ONGC) gave a Rs27bn (\$427m) offshore contract for the Bassein development project to Larsen \& Toubro (L\&T).

i) In February 2016, the board of ONGC approved an investment of Rs. 5,050 crores in Tripura for drilling of wells and creation of surface facilities to produce 5.1 million standard cubic feet per day gas from the state's fields.

j) On 19 July 2017, the Government of India approved the acquisition of Hindustan Petroleum Corporation by ONGC.




\chapter{About Mehsana Asset}

The Mehsana Tectonic Block is a fairly well explored, productive hydrocarbon block of North Cambay Basin. Exploration activity for hydrocarbons by ONGC in the Mehsana Asset commenced in the 1960s and discovered fields are in an advanced stage of exploitation. The Asset has been endowed with a number of oil fields with multilayered pays belonging to Paleocene to middle Miocene age.More than 26 small to medium size oil and gas fields have been established in the Mehsana area of Mehsana-Ahmedabad Tectonic Block.Operational areas for the Mehsana Asset include a Mining Lease(ML) area of 942 Km 2 . In this Asset, 2324 wells have been already
drilled. Further, 35 production installations have already been established to complete the hydrocarbon production, storage and delivery cycle. The Asset currently produces 6100 TPD of crude oil and 5 Lakh m 3 of natural gas on a daily basis.

\vspace{1em}

\noindent Major oil fields within the Asset have been clubbed under six different areas:

\begin{enumerate}
\item Becharaji \& Lanwa (Area-I)

\item Sathal \& Balol (Area-II)

\item Jotana (Area-III)

\item Sobhasan Complex (Area- IV)

\item Nandasan, Linch, Langhnaj, Mansa \& other satellite structures (Area-V), and

\item North Kadi (Area-VI)

\end{enumerate}

\noindent Area I \& II constitute the heavy oil fields of the Mehsana Asset.
The Asset, spread over four districts in North Gujarat, covers:Two talukas of Patan and Ahmedabad Districts Six talukas in Mehsana District,
and One taluka in Gandhinagar District

\section{The Drilling Process}

Drilling operations shall be conducted round-the-clock for 24 hrs. The time taken to drill a
 well depends on the depth of the hydrocarbon bearing formation and the geological conditions.
 ONGC intends to drill wells to a depth range from 1200 to 2000 m. This would typically take ~30 - 35 days
  for each well – however drilling period may increase depending on well depth.
  In general, a 17 1⁄2” hole is drilled from the surface up to a predetermined depth and 13 3/8” surface casing is done to cover fresh water sands, prevent caving, to cover weak zones \& to provide means for attaching well head \& the blowout preventer(BOP).This is followed by drilling of 12 1⁄4” hole and lowering of 9 5/8” intermediate casing depending upon the depth of the well and anticipated problems in drilling the well. The 8 1⁄2” holes is drilled up to the target depth of the well cased with 5 1⁄2” or 7” production casing to isolate the producing zone from the other formations.

\vspace{1em}

In the process of drilling, drilling fluid is used to lift the cutting from the hole to the surface.
Drilling fluid is formulated by earth clay and barites. Various types of bio-degradable polymers are
also added to maintain the specific parameters of the mud. After completion of production casing
the well is tested to determine \& analyze various parameters of producing fluid.

\vspace{1em}

Water based mud, that is ecologically sensitive, will be used and all drilling activities will
be conducted as per the requirements of the Oilfield and Mineral Development Rules, 1
984 as amended till date. Guidelines issued by the Oil Mines Regulation (OMR) will be followed
throughout the drilling process.

\vspace{1em}
The power required for driving the drilling rig, circulation system and for providing lighting
shall be generated by DG sets attached with the rig deployed by ONGC from various
rigs available with ONGC mentioned in the table.. Fuel used in DG sets shall conform to
Bharat Stage IV norms including a sulphur content of <50 mg/kg.


\begin{table}

\centering

\begin{tabular}{|l|l|l|l|}
\hline
Serial no           & \multicolumn{1}{c|}{Rig}                                                                                   & packager    & KVA \\ \hline
\multirow{2}{*}{1.} & \multicolumn{1}{c|}{\multirow{2}{*}{\begin{tabular}[c]{@{}c@{}}IPS-700-V\\ (Mechanical rig)\end{tabular}}} & Sudhir DG   & 500 \\ \cline{3-4}
                    & \multicolumn{1}{c|}{}                                                                                      & Caterpillar & 438 \\ \hline
\multirow{2}{*}{2.} & \multicolumn{1}{c|}{\multirow{2}{*}{\begin{tabular}[c]{@{}c@{}}IPS-700-VI\\ (Mechanical rig)\end{tabular}}} & Jackson     & 500 \\ \cline{3-4}
                    & \multicolumn{1}{c|}{}                                                                                      & Kirloskar   & 380 \\ \hline
\multirow{2}{*}{3.} & \multicolumn{1}{c|}{\multirow{2}{*}{\begin{tabular}[c]{@{}c@{}}IPS-700-VII\\ (Mechanical rig)\end{tabular}}} & Jackson     & 500 \\ \cline{3-4}
                    & \multicolumn{1}{c|}{}                                                                                      & Jeevan DG & 380 \\ \hline
\multirow{2}{*}{4.} & \multicolumn{1}{c|}{\multirow{2}{*}{\begin{tabular}[c]{@{}c@{}}M-900-I\\ (Mechanical rig)\end{tabular}}} & Sudhir DG   & 500 \\ \cline{3-4}
                    & \multicolumn{1}{c|}{}                                                                                      & Kirloskar & 355 \\ \hline
\multirow{2}{*}{5.} & \multicolumn{1}{c|}{\multirow{2}{*}{\begin{tabular}[c]{@{}c@{}}M-750-II\\ (Mechanical rig)\end{tabular}}} & Sudhir DG   & 500 \\ \cline{3-4}
                    & \multicolumn{1}{c|}{}                                                                                      & Sudhir DG & 380 \\ \hline
\multirow{2}{*}{6.} & \multicolumn{1}{c|}{\multirow{2}{*}{\begin{tabular}[c]{@{}c@{}}E-760-XI\\ (Electrial rig)\end{tabular}}} & BHEL   & 1430 \\ \cline{3-4}
                    & \multicolumn{1}{c|}{}                                                                                      & BHEL & 1430 \\ \hline
\multirow{2}{*}{7.} & \multicolumn{1}{c|}{\multirow{2}{*}{\begin{tabular}[c]{@{}c@{}}CH Rig,John-10\\ (Mechanical rig)\end{tabular}}} & Caterpillar   & 500 \\ \cline{3-4}
                    & \multicolumn{1}{c|}{}                                                                                      & Caterpillar & 500 \\ \hline
\multirow{2}{*}{8.} & \multicolumn{1}{c|}{\multirow{2}{*}{\begin{tabular}[c]{@{}c@{}}CH Rig,John-19\\ (Mechanical rig)\end{tabular}}} & Caterpillar   & 500 \\ \cline{3-4}
                    & \multicolumn{1}{c|}{}                                                                                      & caterpillar & 500 \\ \hline
\multirow{2}{*}{9..} & \multicolumn{1}{c|}{\multirow{2}{*}{\begin{tabular}[c]{@{}c@{}}CH Rig,John-23\\ (Mechanical rig)\end{tabular}}} & Caterpillar   & 500 \\ \cline{3-4}
                    & \multicolumn{1}{c|}{}                                                                                      & Caterpillar & 500 \\ \hline

\end{tabular}

\caption{Power Requirement of Various Rigs of ONGC}

\end{table}


\section{Water Requirement}

The drilling operation and maintenance of the drill site facilities have various water requirements.
The most significant of these requirements in terms of quantity is that for mud preparation. T
he other requirements would be for engine cooling, floor / equipment / string washing, sanitation,
fire-fighting storage / make-up and drinking. Water for emergency fire fighting would be
stored in a pit of 200 m 3 capacity and make-up of the same will have to be made on a regular basis.
The requirement of water expected for sanitation and drinking purposes of the workers shall be
insignificantly low in terms of quantity. ONGC has planned to meet the requirement of water at
the drilling site through water supplied by tankers and sourced from nearest ONGC installation.
Since, there is no quality criterion for usage of raw water for the various uses
mentioned above (other than drinking), the tanker water shall be directly used without any treatment.
The potable water requirement shall be met by procuring adequately treated water from off-site locations.


\section{Waste Water Generation}

The drilling operation would generate waste water in the form of wash water due to washing of equipment,
string etc. This waste water along with spill over mud will be diverted to waste water mud pit whose
bottom would be lined with HDPE sheet so as to avoid percolation of water contaminants in the soil.
Approximately 25 m 3 per day of waste water will be discharged in HDPE lined evaporation pit.
The domestic sewage generated from the drill site operations will be treated in a septic tank–soak pit system.
The septic tank is adequately sized to cater to a volumetric capacity of 4–5 m 3 per day.

\section{Air Emissions}

The emissions to the atmosphere from the drilling operations shall be from the diesel engines
and flaring of associated gas during testing operation in case of hydrocarbon is discovered.
In accordance with the Oil Mines Regulations Rules, a flare stack of 9m height will be provided.

\section{Solid and Hazardous Waste Management}

The drilling rig system to be employed for drilling will be equipped for the separation of drill cuttings
and solid materials from the drilling fluid. The drill cuttings, cut by the drill bit, will be removed
from the fluid by the shale shakers (vibrating screens) and centrifuges and transferred to the cuttings
containment area.Once the drilling fluid / mud have been cleaned it will be returned to the fluid tank
and pumped down the drill string again.It is estimated that 104 MT of formation cuttings and 650 m 3 of
drilling mud will be generated in the form of solid waste, during the drilling operation.
Drill cuttings and drilling mud will be disposed off in accordance with the Gazette Notification dated 30th August 2005 - G.S.R 546(E), Section C ‘Guidelines for Disposal of Solid Waste, Drill Cuttings and Drilling Fluids for Offshore and Onshore Drilling
Operation’.

\vspace{1em}

\noindent Under these guidelines:

\vspace{1em}

Drill cuttings separated from Water Based Mud (WBM) will be properly washed and
unusable drilling fluids will be allowed toevaporate in a HDPE lined pit. In case the
drill cuttings have oil and grease level in.



\chapter{Drilling Fluids Services}


\section{Drilling Mud}
	
Drilling Mud or drilling fluid is a critical component in the rotary drilling processes.

The primary functions of drilling fluids are:

\begin{itemize}

\item Remove cuttings from the wellbore.
\item Prevent formation fluids from entering into the wellbore.
\item Maintain wellbore stability
\item Cool and Lubricate the bit
\item Transmit Hydraulic Horsepower to bit

\end{itemize}

The corresponding properties associated with these functions are:
%
%
%
%

\subsection*{Types of Drilling fluids}

The two major types of mud system used are Oil Based Mud which consists of oil 
as a continuous phase and other is Water Based mud which consists of water as 
a continuous phase.Mostly water based mud is used in Mehsana. The following 
mud systems are in continuous use in Mehsana Asset:

1) SPUD Mud
2) KPPA Mud
3) NDDF Mud

\textbf{Spud Mud}

Mud used to drill a well from the surface to a shallow depth .Onshore spud mud
consists of bentonite clay whereas in offshore guar gum and salt gel are used 
in offshore.

\textbf{KPPA Mud:}

It is an abbreviation of KCL, PHPA, polyol mud.KCL and PHPA in the mud acts as 
a shale inhibitor and polyol provides lubricity to the mud.

\textbf{NDDF Mud}

It is generally used near pay zone. It is an abbreviation of Non Damaging 
Destructive Fluid which does not uses barite but polymers as a weighing agent as 
barite is non biodegradable and damages the pay zone permeability and porosity.

\textbf{Field Tests on Drilling Mud}

The mud properties are regularly monitored by mud engineer. These measurements 
will be used to determine if the properties of mud have been deteriorated and 
require treatment or not.

The following tests are used to determine the under mentioned :
\begin{itemize}

\item Mud Density

The density of the drilling mud can be determined with the mud balance shown 
in Figure. The cup of the balance is completely filled with a sample of the
mud and the lid placed firmly on top (some mud should escape through the hole 
in the lid). The balance arm is placed on the base and the rider adjusted until 
the arm is level. The density can be read directly off the graduated scale at 
the left-hand side of the rider.

\item Viscosity

The rheological character of drilling fluids is discussed at length in the 
chapter on Drilling Hydraulics. In general terms however, viscosity is a 
measure of a liquids resistance to flow. Two common methods are used on the rig 
to measure viscosity.

The marsh funnel viscometer is used to make the quickest analysis of the 
viscosity of drilling fluid .However this device only gives change in 
viscosity and does not quantify rheological properties such as yield point 
and plastic viscosity for which we use rotational viscometer.

Rotational viscometer: The multi-rate rotational viscometer is used
to quantify the rheological properties of the drilling mud. The
assessment is made by shearing a sample of the mud, at a series of
prescribed rates and measuring the shear stress on the fluid at these
different rates.


\item Gel Strength

The gel strength of the drilling mud can be thought of as the strength of any internal structures which are formed in the mud when it is static. The gel strength of the mud will provide an indication of the pressure required to initiate flow after the mud has been static for some time. The gel strength of the mud also provides an indication of the suspension properties of the mud and hence its ability to suspend cuttings when the mud is stationary . It is measured using rotational viscometer by measuring viscosity at 3rpm and after 10 seconds.

\item Filtration

The filter cake building properties of mud can be measured by means of a filter press. The following are measured during this test :

1. The rate at which fluid from a mud sample is forced through a filter under specified temperature and pressure.

2. The thickness of the solid residue deposited on the filter paper caused by the loss of fluids.

\item pH Determination

The pH of the mud will influence the reaction of various chemicals and must therefore be closely controlled. The pH test is a measure of the concentration of hydrogen ions in an aqueous solution. This can be done either by pH paper or by a special pH meter.

\end{itemize}
 

\section{\textbf{Rig Visit-RIG-IPS-M700}}

Location: Sobahasan Field Cardwell 6

Date: 1/06/18 to 10/06/18

Rig Name-IPS M700-6

Rig Type-Mechanical Mobile

BOP Type-13 5/8 ” 5000 psi Double Ram \hfill 5 5/8 ” 5000 psi Annular BOP

\underline{Mud Pump}

Power-1250 hP

Power (in KW) - 700KWs

\underline{Well Details}

Well Name-SBLR

Target Depth-1953.4 m

True Vertical Depth- 1920 m

Spud date- 29 th May’18

Well Profile-Inclined Well (Directional Drilling)

Kick Off Point-750 m

Net Horizontal Drift-138.98m

Angle-14.81 degrees

Azimuth- 24.1 degrees North

Pay Zone- 1441 m

\textbf{DAY \hfill 1}

Cementing Job was being done.

500 m depth was drilled

Casing Policy- 2 CP (9 5/8 ” Conductor Casing and 5 1/2 ” Production Casing)

Specific Gravity of Cement- 1.83

Class of Cement- Class G

Thickening Time- 2- 3hours

The image of cementing truck is shown on next page.

DAY 2-6

We were acknowledged with BOP Accumulator


\underline{BOP Accumulator}

An accumulator is a unit used to hydraulically operate Rams BOP,HCR and some hydraulic equipment. There are several high pressure cylinders that store nitrogen or hydraulic fluid or water under pressure for hydraulic activated systems. The primary purpose of this unit is to supply hydraulic power to the BOP stack in order to close or open BOP stack for both normal and emergency situation. Stored hydraulic fluid in the accumulator provides hydraulic power to close BOP in well control operation so that the kick is minimized.The BOP accumulator was charged at 3000 psi with nitrogen. The minimum pressure required to close the BOP was 1200 psi.The ratings of both Annular Preventer and Shear and Blind Ram were 5000 psi.The BOP at the RIG is pressure tested after every 21 days or whenever the rig is set up at new site and BOP is installed whichever is the least. Pressure test is done using water if there will be leakage the water will leak from that points.The BOP is also function tested whether it is being open or closed when given command to open or close.


\noindent Then we were briefed about the solid control equipments.Solid Control Equipments The solid control equipments consist of:

\begin{itemize}

\item Shale Shaker
\item Desander
\item Desilter
\item Degasser

\end{itemize}

 


\end{document}
