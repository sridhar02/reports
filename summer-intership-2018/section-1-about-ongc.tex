\chapter{About ONGC}

Oil and Natural Gas Corporation Limited (ONGC) is an Indian multinational oil and gas company earlier headquartered in Dehradun, Uttarakhand, India. As a Corporation, its registered office is now at Deendayal Uurja Bhavan, Vasant Kunj, New Delhi (110070) India. It is a Public Sector Undertaking (PSU) of the
Government of India, under the administrative control of the Ministry of Petroleum and Natural Gas. It is India's largest oil and gas exploration and production company. It produces around 70\% of India's crude oil (equivalent to around 30\% of the country's total demand) and around 62\% of its natural gas.

\section{History of ONGC}

\begin{enumerate}

\item Foundation to 1961

\begin{enumerate}[(a)]

\item Before the independence of India in 1947, the Assam Oil Company in the north-eastern 
and Attock Oil Company in north-western part of the undivided India were the only oil producing
companies, with minimal exploration input. The major part of Indian sedimentary basins was 
deemed to be unfit for development of oil and gas resources.

\item After independence, the Central Government of India realized the importance of oil and gas 
for rapidindustrial development and its strategic role in defense.Consequently, while framing 
the Industrial Policy Statement of 1948, the development of petroleum industry in the country 
was considered to be of utmost necessity.

\item In 1955, Government of India decided to develop the oil and natural gas resources 
in the various regions of the country as part of the Public Sector development. 
With this objective, an Oil and Natural Gas Directorate was set up towards the end of 1955,
 as a subordinate office under the then Ministry of Natural Resources and Scientific Research. 
 The department was constituted with a nucleus of geoscientists from the Geological Survey of India.

\item Soon, after the formation of the Oil and Natural Gas Directorate, it became apparent that it would 
not be possible for the Directorate with its limited financial and administrative powers as subordinate 
office of the Government, to function efficiently. So in August, 1956, the Directorate was raised to the
status of a commission with enhanced powers, although it continued to be under the government. 
In October 1959, the Commission was converted into a statutory body by an act of the Indian Parliament, 
which enhanced powers of the commission further. The main functions of the Oil and Natural Gas Commission 
subject to the provisions of the Act, were "to plan, promote, organize and implement programs for 
development of Petroleum Resources and the production and sale of petroleum and petroleum products produced 
by it, and to perform such other functions as the Central Government may, 
from time to time, assign to it ". The act further outlined the activities and steps to be taken 
by ONGC in fulfilling its mandate.

\end{enumerate}


\item From 1961 to present

\begin{enumerate}[(a)]

\item Since its inception, ONGC has been instrumental in transforming the country's limited upstream sector into a large viable playing field, with its activities spread throughout India and significantly in overseas territories. In the inland areas, ONGC not only found new resources in Assam but also established new oil province in Cambay basin (Gujarat), while adding new petroliferous areas in the Assam-Arakan Fold Belt and East coast basins (both onshore and offshore).

\item ONGC became a publicly held company in February 1994, with 20\% of its equity were sold to the public and eighty percent retained by the Indian government. At the time, ONGC employed 48,000 people and had reserves and surpluses worth ₹104.34 billion, in addition to its intangible assets. The corporation's net worth of ₹107.77 billion was the largest of any Indian company .

\item In 2003, ONGC Videsh Limited (OVL), the division of ONGC concerned with its foreign assets, acquired Talisman Energy's 25\% stake in the Greater Nile Oil project.

\item In 2006, a commemorative coin set was issued to mark the 50th anniversary of the founding of ONGC, making it only the second Indian company (State Bank of India being the first) to have such a coin issued in its honor.

\item In 2011, ONGC applied to purchase 2000 acres of land at Dahanu to process offshore gas. ONGC Videsh, along with Statoil ASA (Norway) and Repsol SA (Spain), has been engaged in deep-water drilling off the northern coast of Cuba in 2012. On 11 August 2012, ONGC announced that it had made a large oil discovery in the D1 oilfield off the west coast of India, which will help it to raise the output of the field from around 12,500 barrels per day (bpd) to a peak output of 60,000 bpd.


\begin{table}[H]

\centering

\begin{tabular}{|l|l|lll}
\cline{1-2} 
\textbf{Product}     & \textbf{Revenue} &  &  &  \\ \cline{1-2}
Crude Oil            & 526.38           &  &  &  \\ \cline{1-2}
Gas                  & 168.88           &  &  &  \\ \cline{1-2}
LPG                  & 31.48            &  &  &  \\ \cline{1-2}
Naptha               & 36.80            &  &  &  \\ \cline{1-2}
C2-C3                & 13.44            &  &  &  \\ \cline{1-2}
Others               & 1.59             &  &  &  \\ \cline{1-2}
Adjustments          & -32.74           &  &  &  \\ \cline{1-2}
Total                & 825.52           &  &  &  \\ \cline{1-2}
\multicolumn{2}{|l|}{Rs.825.52 billion} &  &  &  \\ \cline{1-2}
\end{tabular}

\caption{Product-wise revenue breakup for FY 2016–17 (₹ billion)}
\end{table}

\vspace{1em}

\item In November 2012, OVL agreed to acquire ConocoPhillips' 8.4\% stake in the Kashagan oilfield in Kazakhstan for around US\$5 billion, in ONGC's largest acquisition to date. The acquisition is subject to the approval of the governments of Kazakhstan and India and also to other partners in the Caspian Sea field waiving their pre-emption rights.

\item In January 2014, OVL and Oil India completed the acquisition of Videocon Group's ten percent stake in a Mozambican gas field for a total of \$2.47 billion.

\item In June 2015, Oil and Natural Gas Corporation (ONGC) gave a Rs27bn (\$427m) offshore contract for the Bassein development project to Larsen \& Toubro (L\&T).

\item In February 2016, the board of ONGC approved an investment of Rs. 5,050 crores in Tripura for drilling of wells and creation of surface facilities to produce 5.1 million standard cubic feet per day gas from the state's fields.

\item On 19 July 2017, the Government of India approved the acquisition of Hindustan Petroleum Corporation by ONGC.

\vspace{1em}

\end{enumerate}

\end{enumerate}

\noindent Table 1.1 shows the Product-wise revenue breakup for FY 2016–17 (₹ billion) which is shown above.

\vspace{2em}


\noindent ONGC Videsh is a wholly owned subsidiary of Oil and Natural Gas Corporation Limited (ONGC), 
the National Oil Company of India, and is India’s largest international oil and gas Company. 
ONGC Videsh has participation in 41 projects in 20 countries namely Azerbaijan, Bangladesh, Brazil, 
Colombia, Iraq, Israel, Iran, Kazakhstan, Libya, Mozambique, Myanmar, Namibia, Russia, South Sudan, Sudan,
Syria, United Arab Emirates, Venezuela, Vietnam and New Zealand. ONGC Videsh maintains 
a balanced portfolio of 15 producing, 4 discovered/under development, 18 exploratory and 4 pipeline projects. 
The Company currently operates/ jointly operates 21 projects. ONGC Videsh had total oil and gas reserves (2P) 
of about 711 MMTOE as on April 1, 2018.

\vspace{2em}

\noindent ONGC has implemented globally recognized QHSE management systems conforming to requirements of ISO 9001, 
OHSAS 18001 and ISO 14001 at ONGC facilities and certified by reputed certification agencies at all its 
operational units. Corporate guidelines on incident reporting, investigation and monitoring of recommendations
has been developed and implemented for maintaining uniformity throughout the organization in line with 
international practice.

\vspace{2em}

\noindent Corporate Disaster Management Plan and guidelines have been developed for uniform disaster management 
all across ONGC. ONGC has also developed Occupational Health physical fitness criteria for employees 
deployed for offshore operations. Occupational Health module has now been populated on SAP system.
