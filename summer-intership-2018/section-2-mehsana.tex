\chapter{About Mehsana Asset}

The Mehsana Tectonic Block is a fairly well explored, productive hydrocarbon block of North Cambay Basin. Exploration activity for hydrocarbons by ONGC in the Mehsana Asset commenced in the 1960s and discovered fields are in an advanced stage of exploitation. The Asset has been endowed with a number of oil fields with multilayered pays belonging to Paleocene to middle Miocene age.More than 26 small to medium size oil and gas fields have been established in the Mehsana area of Mehsana-Ahmedabad Tectonic Block.Operational areas for the Mehsana Asset include a Mining Lease(ML) area of 942 Km 2 . In this Asset, 2324 wells have been already
drilled. Further, 35 production installations have already been established to complete the hydrocarbon production, storage and delivery cycle. The Asset currently produces 6100 TPD of crude oil and 5 Lakh m 3 of natural gas on a daily basis.

Major oil fields within the Asset have been clubbed under six different areas:

1. Becharaji \& Lanwa (Area-I)

2. Sathal \& Balol (Area-II)

3. Jotana (Area-III)

4. Sobhasan Complex (Area- IV)

5. Nandasan, Linch, Langhnaj, Mansa \& other satellite structures (Area-V), and

6. North Kadi (Area-VI)

Area I \& II constitute the heavy oil fields of the Mehsana Asset.

The Asset, spread over four districts in North Gujarat, covers:

Two talukas of Patan and Ahmedabad Districts Six talukas in Mehsana District,and One taluka in Gandhinagar District

\section{\textbf{The Drilling Process}}

Drilling operations shall be conducted round-the-clock for 24 hrs. The time taken to drill a well depends on the depth of the hydrocarbon bearing formation and the geological conditions.ONGC intends to drill wells to a depth range from 1200 to 2000 m. This would typically take ~30 - 35 days for each well – however drilling period may increase depending on well depth.In general, a 17 1⁄2” hole is drilled from the surface up to a predetermined depth and 13 3/8” surface casing is done to cover fresh water sands, prevent caving, to cover weak zones \& to provide means for attaching well head \& the blowout preventer(BOP).This is followed by drilling of 12 1⁄4” hole and lowering of 9 5/8” intermediate casing depending upon the depth of the well and anticipated problems in drilling the well. The 8 1⁄2” holes is drilled up to the target depth of the well cased with 5 1⁄2” or 7” production casing to isolate the producing zone from the other formations.

In the process of drilling, drilling fluid is used to lift the cutting from the hole to the surface. Drilling fluid is formulated by earth clay and barites. Various types of bio-degradable polymers are also added to maintain the specific parameters of the mud. After completion of production casing the well is tested to determine \& analyze various parameters of producing fluid.

Water based mud, that is ecologically sensitive, will be used and all drilling activities will be conducted as per the requirements of the Oilfield and Mineral Development Rules, 1984 as amended till date. Guidelines issued by the Oil Mines Regulation (OMR) will be followed throughout the drilling process.

The power required for driving the drilling rig, circulation system and for providing lighting shall be generated by DG sets attached with the rig deployed by ONGC from various rigs available with ONGC mentioned in the table.. Fuel used in DG sets shall conform to Bharat Stage IV norms including a sulphur content of <50 mg/kg.


\newpage

\begin{table}
\centering
\caption{Power Distribution Table}

\begin{tabular}{|l|l|l|l|}
\hline
Serial no           & \multicolumn{1}{c|}{Rig}                                                                                   & packager    & KVA \\ \hline
\multirow{2}{*}{1.} & \multicolumn{1}{c|}{\multirow{2}{*}{\begin{tabular}[c]{@{}c@{}}IPS-700-V\\ (Mechanical rig)\end{tabular}}} & Sudhir DG   & 500 \\ \cline{3-4} 
                    & \multicolumn{1}{c|}{}                                                                                      & Caterpillar & 438 \\ \hline
\multirow{2}{*}{2.} & \multicolumn{1}{c|}{\multirow{2}{*}{\begin{tabular}[c]{@{}c@{}}IPS-700-V\\ (Mechanical rig)\end{tabular}}} & Sudhir DG   & 500 \\ \cline{3-4} 
                    & \multicolumn{1}{c|}{}                                                                                      & Caterpillar & 438 \\ \hline
\multirow{2}{*}{3.} & \multicolumn{1}{c|}{\multirow{2}{*}{\begin{tabular}[c]{@{}c@{}}IPS-700-V\\ (Mechanical rig)\end{tabular}}} & Sudhir DG   & 500 \\ \cline{3-4} 
                    & \multicolumn{1}{c|}{}                                                                                      & Caterpillar & 438 \\ \hline
\multirow{2}{*}{4.} & \multicolumn{1}{c|}{\multirow{2}{*}{\begin{tabular}[c]{@{}c@{}}IPS-700-V\\ (Mechanical rig)\end{tabular}}} & Sudhir DG   & 500 \\ \cline{3-4} 
                    & \multicolumn{1}{c|}{}                                                                                      & Caterpillar & 438 \\ \hline
\multirow{2}{*}{1.} & \multicolumn{1}{c|}{\multirow{2}{*}{\begin{tabular}[c]{@{}c@{}}IPS-700-V\\ (Mechanical rig)\end{tabular}}} & Sudhir DG   & 500 \\ \cline{3-4} 
                    & \multicolumn{1}{c|}{}                                                                                      & Caterpillar & 438 \\ \hline
\multirow{2}{*}{1.} & \multicolumn{1}{c|}{\multirow{2}{*}{\begin{tabular}[c]{@{}c@{}}IPS-700-V\\ (Mechanical rig)\end{tabular}}} & Sudhir DG   & 500 \\ \cline{3-4} 
                    & \multicolumn{1}{c|}{}                                                                                      & Caterpillar & 438 \\ \hline
\multirow{2}{*}{1.} & \multicolumn{1}{c|}{\multirow{2}{*}{\begin{tabular}[c]{@{}c@{}}IPS-700-V\\ (Mechanical rig)\end{tabular}}} & Sudhir DG   & 500 \\ \cline{3-4} 
                    & \multicolumn{1}{c|}{}                                                                                      & Caterpillar & 438 \\ \hline
\multirow{2}{*}{1.} & \multicolumn{1}{c|}{\multirow{2}{*}{\begin{tabular}[c]{@{}c@{}}IPS-700-V\\ (Mechanical rig)\end{tabular}}} & Sudhir DG   & 500 \\ \cline{3-4} 
                    & \multicolumn{1}{c|}{}                                                                                      & Caterpillar & 438 \\ \hline
\multirow{2}{*}{1.} & \multicolumn{1}{c|}{\multirow{2}{*}{\begin{tabular}[c]{@{}c@{}}IPS-700-V\\ (Mechanical rig)\end{tabular}}} & Sudhir DG   & 500 \\ \cline{3-4} 
                    & \multicolumn{1}{c|}{}                                                                                      & Caterpillar & 438 \\ \hline
\multirow{2}{*}{1.} & \multicolumn{1}{c|}{\multirow{2}{*}{\begin{tabular}[c]{@{}c@{}}IPS-700-V\\ (Mechanical rig)\end{tabular}}} & Sudhir DG   & 500 \\ \cline{3-4} 
                    & \multicolumn{1}{c|}{}                                                                                      & Caterpillar & 438 \\ \hline

\multirow{2}{*}{2.} & \multirow{2}{*}{}                                                                                          &             &     \\ \cline{3-4} 
                    &                                                                                                            &             &     \\ \hline
\end{tabular}
\end{table}


\section{\textbf{Water Requirement}}

The drilling operation and maintenance of the drill site facilities have various water requirements. The most significant of these requirements in terms of quantity is that for mud preparation. The other requirements would be for engine cooling, floor / equipment / string washing, sanitation, fire-fighting storage / make-up and
drinking. Water for emergency fire fighting would be stored in a pit of 200 m 3 capacity and make-up of the same will have to be made on a regular basis.The requirement of water expected for sanitation and drinking purposes of the workers shall be insignificantly low in terms of quantity. ONGC has planned to meet the requirement of water at the drilling site through water supplied by tankers and sourced from nearest ONGC installation. Since, there is no quality criterion for usage of raw water for the various uses mentioned above (other than drinking), the tanker water shall be directly used without any treatment.The potable water requirement shall be met by procuring adequately treated water from off-site locations.

\section{\textbf{Waste Water Generation}}

The drilling operation would generate waste water in the form of wash water due to washing of equipment, string etc. This waste water along with spill over mud will be diverted to waste water mud pit whose bottom would be lined with HDPE sheet so as to avoid percolation of water contaminants in the soil. Approximately 25 m 3 per day of waste water will be discharged in HDPE lined evaporation pit. The domestic sewage generated from the drill site operations will be treated in a septic tank–soak pit system. The septic tank is adequately sized to cater to a volumetric capacity of 4–5 m 3 per day.

\section{\textbf{Air Emissions}}

The emissions to the atmosphere from the drilling operations shall be from the diesel engines and flaring of associated gas during testing operation in case of hydrocarbon is discovered. In accordance with the Oil Mines Regulations Rules, a flare stack of 9m height will be provided.

\section{\textbf{Solid and Hazardous Waste Management}}

The drilling rig system to be employed for drilling will be equipped for the separation of drill cuttings and solid materials from the drilling fluid. The drill cuttings, cut by the drill bit, will be removed from the fluid by the shale shakers (vibrating screens) and centrifuges and transferred to the cuttings containment area.Once the drilling fluid / mud have been cleaned it will be returned to the fluid tank and pumped down the drill string again.It is estimated that 104 MT of formation cuttings and 650 m 3 of drilling mud will be generated in the form of solid waste, during the drilling operation.

Drill cuttings and drilling mud will be disposed off in accordance with the Gazette Notification dated 30th August 2005 - G.S.R 546(E), Section C ‘Guidelines for Disposal of Solid Waste, Drill Cuttings and Drilling Fluids for Offshore and Onshore Drilling
Operation’. 

Under these guidelines:

Drill cuttings separated from Water Based Mud (WBM) will be properly washed and unusable drilling fluids will be allowed toevaporate in a HDPE lined pit. In case the drill cuttings have oil and grease level in.

