\chapter{Drilling Fluids Services}


\section{Drilling Mud}
	
Drilling Mud or drilling fluid is a critical component in the rotary drilling processes.

The primary functions of drilling fluids are:

\begin{itemize}

\item Remove cuttings from the wellbore.
\item Prevent formation fluids from entering into the wellbore.
\item Maintain wellbore stability
\item Cool and Lubricate the bit
\item Transmit Hydraulic Horsepower to bit

\end{itemize}

The corresponding properties associated with these functions are:
%
%
%
%

\subsection*{Types of Drilling fluids}

The two major types of mud system used are Oil Based Mud which consists of oil 
as a continuous phase and other is Water Based mud which consists of water as 
a continuous phase.Mostly water based mud is used in Mehsana. The following 
mud systems are in continuous use in Mehsana Asset:

1) SPUD Mud
2) KPPA Mud
3) NDDF Mud

\textbf{Spud Mud}

Mud used to drill a well from the surface to a shallow depth .Onshore spud mud
consists of bentonite clay whereas in offshore guar gum and salt gel are used 
in offshore.

\textbf{KPPA Mud:}

It is an abbreviation of KCL, PHPA, polyol mud.KCL and PHPA in the mud acts as 
a shale inhibitor and polyol provides lubricity to the mud.

\textbf{NDDF Mud}

It is generally used near pay zone. It is an abbreviation of Non Damaging 
Destructive Fluid which does not uses barite but polymers as a weighing agent as 
barite is non biodegradable and damages the pay zone permeability and porosity.

\textbf{Field Tests on Drilling Mud}

The mud properties are regularly monitored by mud engineer. These measurements 
will be used to determine if the properties of mud have been deteriorated and 
require treatment or not.

The following tests are used to determine the under mentioned :
\begin{itemize}

\item Mud Density

The density of the drilling mud can be determined with the mud balance shown 
in Figure. The cup of the balance is completely filled with a sample of the
mud and the lid placed firmly on top (some mud should escape through the hole 
in the lid). The balance arm is placed on the base and the rider adjusted until 
the arm is level. The density can be read directly off the graduated scale at 
the left-hand side of the rider.

\item Viscosity

The rheological character of drilling fluids is discussed at length in the 
chapter on Drilling Hydraulics. In general terms however, viscosity is a 
measure of a liquids resistance to flow. Two common methods are used on the rig 
to measure viscosity.

The marsh funnel viscometer is used to make the quickest analysis of the 
viscosity of drilling fluid .However this device only gives change in 
viscosity and does not quantify rheological properties such as yield point 
and plastic viscosity for which we use rotational viscometer.

Rotational viscometer: The multi-rate rotational viscometer is used
to quantify the rheological properties of the drilling mud. The
assessment is made by shearing a sample of the mud, at a series of
prescribed rates and measuring the shear stress on the fluid at these
different rates.


\item Gel Strength

The gel strength of the drilling mud can be thought of as the strength of any internal structures which are formed in the mud when it is static. The gel strength of the mud will provide an indication of the pressure required to initiate flow after the mud has been static for some time. The gel strength of the mud also provides an indication of the suspension properties of the mud and hence its ability to suspend cuttings when the mud is stationary . It is measured using rotational viscometer by measuring viscosity at 3rpm and after 10 seconds.

\item Filtration

The filter cake building properties of mud can be measured by means of a filter press. The following are measured during this test :

1. The rate at which fluid from a mud sample is forced through a filter under specified temperature and pressure.

2. The thickness of the solid residue deposited on the filter paper caused by the loss of fluids.

\item pH Determination

The pH of the mud will influence the reaction of various chemicals and must therefore be closely controlled. The pH test is a measure of the concentration of hydrogen ions in an aqueous solution. This can be done either by pH paper or by a special pH meter.

\end{itemize}
 