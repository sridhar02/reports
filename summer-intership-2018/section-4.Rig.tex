\chapter{Rig Visit-RIG-IPS-M700}

\underline{Location:} 

Sobahasan Field Cardwell 6

Rig Name-IPS M700-6

Rig Type-Mechanical Mobile

BOP Type-13 5/8 ” 5000 psi Double Ram and 5 5/8 ” 5000 psi Annular BOP

\vspace{1em}

\noindent{\underline{Mud Pump}}

Power-1250 hP

Power (in KW) - 700KWs

\vspace{1em}

\noindent{\underline{Well Details}}

Well Name-SBLR

Target Depth-1953.4 m

True Vertical Depth- 1920 m

Spud date- 29 th May’18

Well Profile-Inclined Well (Directional Drilling)

Kick Off Point-750 m

Net Horizontal Drift-138.98m

Angle-14.81 degrees

Azimuth- 24.1 degrees North

Pay Zone- 1441 m

\noindent{\textbf{DAY 1:}}

Cementing Job was being done.

500 m depth was drilled

Casing Policy- 2 CP (9 5/8 ” Conductor Casing and 5 1/2 ” Production Casing)

Specific Gravity of Cement- 1.83

Class of Cement- Class G

Thickening Time- 2- 3hours

The image of cementing truck is shown on next page.

\noindent{\textbf{DAY 2-6:}}

We were acknowledged with BOP Accumulator


\noindent{\underline{\textbf{BOP Accumulator}}}

An accumulator is a unit used to hydraulically operate Rams BOP,HCR and some hydraulic equipment. There are several high pressure cylinders that store nitrogen or hydraulic fluid or water under pressure for hydraulic activated systems. The primary purpose of this unit is to supply hydraulic power to the BOP stack in order to close or open BOP stack for both normal and emergency situation. Stored hydraulic fluid in the accumulator provides hydraulic power to close BOP in well control operation so that the kick is minimized.The BOP accumulator was charged at 3000 psi with nitrogen. The minimum pressure required to close the BOP was 1200 psi.The ratings of both Annular Preventer and Shear and Blind Ram were 5000 psi.The BOP at the RIG is pressure tested after every 21 days or whenever the rig is set up at new site and BOP is installed whichever is the least. Pressure test is done using water if there will be leakage the water will leak from that points.The BOP is also function tested whether it is being open or closed when given command to open or close.


\noindent Then we were briefed about the solid control equipments.Solid Control Equipments The solid control equipments consist of:

\begin{itemize}

\item Shale Shaker
\item Desander
\item Desilter
\item Degasser

\end{itemize}


\noindent 1.Shale Shaker: It separates big cuttings from mud by the mechanism of vibration. 
It consists of hopper which acts as aplatform for the shaker and collection 
pan for the fluid processed by the shale screens, also known as underflow 
and other component is the feeder whose function is to collect drilling fluid
before it is processed by the shaker, it comes in different shapes
and sizes to accommodate the needs of mud system.

\vspace{1em}

\noindent 2.Desander- It is a hydro cyclone device that removes large drill
solids from the whole mud system. The Desander should be
located downstream of the shale shaker and degasser .A volume of
mud is pumped into wide upper section of hydro cyclone at an
angle roughly tangent to its circumference. As the mud flows
around and gradually down the inside of the cone shape, solids are
separated from the liquid by the centrifugal forces. The solids
continue around and down until they exit the bottom of the hydro
cyclone and are discarded

\vspace{1em}

\noindent 3. Desilter-It is a hydro cyclone much like a Desander except that its
design incorporates a greater number of small cores. As with the
Desilter its purpose is to remove unwanted solids from the mud
system. The smaller cones allow the Desilter to efficiently remove
smaller diameter drill solids than a Desander does.

\vspace{1em}

\noindent 4. Degasser-They are the devices to mechanically remove entrained
gases in the mud. It works on the principle that if the mud is
suddenly sucked from its head known as degasser head gas beinglighter rises
 up and the mud is taken out from the bottom of the
degasser.

\vspace{1em}

\noindent{\textbf{DAY 7-10:}}

On day 7 we were briefed about the mock fire drill on the Rig by the
Rig in charge Mr.Sharma which took place in the following sequence:

\vspace{0.5em}

\noindent 1. An alarm will ring and all staff on the rig will run towards a pre
decided spot except the driller working on the derrick floor.

\noindent 2.Then the number of heads will be counted and the rescue team
 consisting of 2 mechanical staff personnel, 2 electric staff
 personnel , 2 survey man, 3 person with hose and one
 supervisor.

\noindent 3. The mechanical men go on the pump and shut off the pump.

\noindent 4. The electrical men shut off the generators.

\noindent 5. 2 survey men survey the area and rescue any person which
   might be stuck there.

\noindent 6. 3 people with water hose make water umbrella above the fire
and the flow is controlled by the supervisor.

\noindent 7. Fire station is called afterwards.


On day 8 we visited the electrical section and saw the generators
which gave power to the rig.

On day 9 we were taught by casing integrity or hermitical test. It
refers to closed cycle pressure testing of casing of the wells. 

This iscarried out by pumping water at steady rate to detect the leakage
before handling the well for production testing.
 
