\chapter{Cementing}
	
Cementing operations consist in placing an appropriate cement slurry in 
the annulus between the walls of the hole and the casing that has been run in.
There are several types of cementing jobs and each one meets a particular need.

\vspace{1em}

\textbf{PURPOSE:}

\begin{itemize}

\item Isolating a producing formation from adjacent beds.  
\item Securing the casing mechanically to the borehole walls.
\item Protecting casing from corrosion by fluids contained in the beds that have been drilled.
\item Providing a leak-proof base for safety and control equipment that is installed on the wellhead.
\item Injecting extra cement through the perforations in the casing to consolidate or repair the primary cementing job.
\item Sealing off a depleted productive layer.
\item Isolating a bed from adjacent zones to reduce the per cent of water or gas in oil production.
\item Seal off water influxes.
\item Plug up lost circulation zones.
\item Serve as the basis of a side track.

\end{itemize}


Primary Cementation:

It occurs in 4 stages:

1. The first stage comprises of displacing top plug with pre flush
which is water. The bottom plug cleans up the well by
displacing mud and water occupies its place.

2. Then cement slurry is followed after pre flush is added.

3. Once the pre calculated slurry volume has been added, top plug
is displaced again by spacer (water).

4. Pressure is applied in this process and once the top plug rests
upon bottom plug the bottom plug breaks and tremendous
increase in pressure is noted which tells us the cement job.

CEMENTING THE WELL

\vspace{1em}

After casing, or steel pipe, is run into the well, an L-shaped cementing 
head is fixed to the top of the wellhead to receive the slurry from the pumps. 
Two wiper plugs, or cementing plugs, that sweep the
inside of the casing and prevent mixing: the bottom plug and the top plug.
Keeping the drilling fluids from mixing with the cement slurry, 
the bottom plug is introduced into the well, and cement slurry is pumped into the well behind it. 
The bottom plug is then caught just above the bottom of the wellbore by the float collar, 
which functions as a one-way valve allowing the cement slurry to enter the well.
Then the pressure on the cement being pumped into the well is increased until a diaphragm is broken within the bottom plug,
 permitting the slurry to flow through it and up the outside of the casing string.


%figure of the top plug and bottom plug have to included 

After the proper volume of cement is pumped into the well, a top plug is pumped into 
the casing pushing the remaining slurry through the bottom plug.
 Once the top plug reaches the bottom plug, the pumps are turned off, and the cement is allowed to set.
The amount of time it takes cement to harden is called thickening time or pump ability time. 
For setting wells at deep depths, under high temperature or pressure, as well as in corrosive environments,
 special cements can be employed.


    