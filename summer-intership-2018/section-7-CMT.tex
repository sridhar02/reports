\chapter{Crises Management Team (CMT) Section}

Crisis management is the process by which an organization which
deals with a disruptive and unexpected action that threatens to harm
the organization or its stakeholders. The study of crisis management
originated with large scale industrial and environmental disaster in
1980’s. It is considered to be the most important process in public
relations.

\vspace{1em}


\noindent Three elements are common to crises:

\begin{enumerate}

\item Threat to the organization

\item The element of surprise

\item Short decision time

\end{enumerate}


\vspace{1em}

In contrast to risk management, which involves assessing the
potential threat and finding the best ways to avoid the threats, crisis
management involves dealing with threats before, during and after
they have occurred. It is a discipline within a broader context of
management consisting of skills and techniques required to identify,assess understand and cope with a serious situation especially from
the moment it first occurred to the point that recovery procedure start.

\vspace{1em}

Crisis management only has to happen because systems fail, so policies and procedures must do all 
they can to help to prevent incidents from happening in the first place. Effective contingency planning,
implemented by a strong, capable and robust team, should minimise their impact if they do.

\vspace{2em}

Staffing specialists must ensure they have an in-depth understanding of the health and 
safety issues faced by the sector. In this way they can help support their clients and 
help ensure that if the worst happens, the right people are in place to deal with any consequences.

