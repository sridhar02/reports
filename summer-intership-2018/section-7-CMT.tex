\chapter{Crises Management Team (CMT) Section}

Crisis management is the process by which an organization which
deals with a disruptive and unexpected action that threatens to harm
the organization or its stakeholders. The study of crisis management
originated with large scale industrial and environmental disaster in
1980’s. It is considered to be the most important process in public
relations.

\vspace{1em}


Three elements are common to crises:


a) Threat to the organization

\vspace{1em}

b) The element of surprise

\vspace{1em}


c) Short decision time


\vspace{1em}


In contrast to risk management, which involves assessing the
potential threat and finding the best ways to avoid the threats, crisis
management involves dealing with threats before, during and after
they have occurred. It is a discipline within a broader context of
management consisting of skills and techniques required to identify,assess understand and cope with a serious situation especially from
the moment it first occurred to the point that recovery procedure start.